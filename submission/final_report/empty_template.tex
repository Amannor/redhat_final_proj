\documentclass{article}
\usepackage[utf8]{inputenc}

\title{Project name - company name}
\author{Student name (ID. 123456789)   X2}
\date{Submitted as final project report for Projects with the industry workshop, IDC, 2020}

\usepackage{natbib}
\usepackage{graphicx}

\begin{document}

\maketitle

\section{Introduction}

Provide some background on your project. 

\subsection{Related Works}
Please provide some details on existing works or models you referred to. The links to the external sources should appear in the reference section.  


\section{Solution}
\subsection{General approach}
Describe your general approach to the problem. How you want to solve it, what alternatives you plan to try and why. 

\subsection{Data set}
Detail what data set did you use. Its size. Where and how did you find it. Problems you had with the data set and  any other relevant information.

\subsection{Design}
Provide some general information about your code, platform, how long it took you to train it, technical challenges you had, etc. In case initial dataset augmentation/engineering had to be done please elaborate on that aspect.

\section{Experimental results}
Provide information about your experimental settings (split, measurement metric..). What alternatives did you measure? Make sure this part is clear to understand, provide as much details as possible. It is better to report on results with tables and figures.

\section{Discussion}
Provide some final words and summarize what you have found from running the experiments you described above. Provide some high level insights.

\section{What you have learned}
Detail what you have learned while working on the project in the aspects of new skills, working with a company, planning and delivering a project and 

\section{Potential future work}
Detail what other possibilities/fields could be explored in the problem you were working on. Please be accurate and informative since future students might want to pick up from where you stopped.

\section{Code}

Please provide a link to your code repository. It can be either a Github repository either a colab notebook with additional links for the data.
Good luck!!

Good luck!!
\bibliographystyle{plain}
\bibliography{references}
\end{document}
